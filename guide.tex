\documentclass{article}
\usepackage{hyperref}

\title{Guide for Martian Climate Project}
\author{Karim El-Sharkawy of Purdue Mathematics}
\date{}

\begin{document}

\maketitle

\section*{Introduction}
This document provides a simple map for anyone wanting to familiarize themselves with this project. This can be used for review or as a guide. While I was working on my project alone, I was in a team with Audrey Durham and Benjamin Carpenter which was supervised by Professor Lei Wang (\url{website}) and graduate student Zhaoyu Liu of Purdue Earth, Atmospheric, and Planetary Sciences (EAPS).

\section*{Principle topics}
A great resource for this is the AMS glossary, which allows you to search up any atmospheric/planetary science word and it’ll tell you the definition of it \\
\href{https://glossary.ametsoc.org}{Glossary of Meteorology} ← Underline indicates a clickable link \\
\href{https://www.unidata.ucar.edu/software/netcdf/docs/index.html}{NetCDF (Unidata)} (Tutorial) \\
\href{https://www.youtube.com/results?search_query=Panoply+Tutorial}{Panoply} (Video) \\
You can also read NetCDF files through Python/R, but it’s better to use Panoply \\
\href{https://www.weather.gov/jetstream/rossby}{Rossby Waves (NOAA)} (Wikipedia) \\
Annular Modes (AMW) \\
Also, look into Southern Annular Modes (SAMs) \\
\href{https://en.wikipedia.org/wiki/Eddy_(fluid_dynamics)}{Eddies (Wikipedia)}

This is a tight-knit community, so knowing and contacting the big names in this field is crucial. After knowing the big names, you’ll come across some of their research and books, which will have many technical words you’ll need to understand to get a good understanding of the paper. This would be a great time to use the AMS Glossary.

\section*{The big names}
Adam Showman (\href{https://www.lpl.arizona.edu/faculty/adam-showman}{Page}) \\
Read this great \href{https://arxiv.org/pdf/0911.3170.pdf}{paper} \\
Juan M. Lora (\href{https://people.earth.yale.edu/profile/juan-lora/about}{Page}) \\
Michael Battalio (\href{https://battalio.com/}{Website}) \\
Wanying Kang (\href{https://wanyingkang.com/}{website}) \\
\url{https://eapsweb.mit.edu/people/wanying}

\section*{Important notes}
\href{https://atmos.washington.edu/~dennis/552_Notes_ftp.html}{Important Notes}

Although I didn’t finish my project, I had many interesting thoughts that I would like to share here, in case it interests anyone! What are the similarities of the Rossby waves on mars compared to Earth? Can we somehow, just from knowing the Rossby waves on a planet, derive other properties that don’t seem connected?

\section*{Karim Mohamed El-Sharkawy Research Document}
\subsection*{Summary}
This document outlines the comprehensive research activities and objectives of Karim Mohamed El-Sharkawy, a researcher focused on Martian and planetary climates. The document includes detailed plans, meeting notes, tasks, and reflections related to his research.

\subsection*{Research Objectives}
\textbf{Short-term Goals:}
\begin{itemize}
    \item Improve understanding of Mars' climate and planetary climates.
    \item Gain insights into research methodologies.
    \item Study annular modes and quantify them using EMARS and MACDA datasets.
    \item Read relevant literature, including Joe Michael Battalio’s work on annular modes in the atmospheres of Mars and Titan.
\end{itemize}

\textbf{Long-term Goals:}
\begin{itemize}
    \item Contribute to atmospheric science by comparing the climates of different planets to Earth's climate.
    \item Develop a research career in planetary climates.
\end{itemize}

\subsection*{Detailed Activities and Tasks}
\begin{itemize}
    \item \textbf{Meetings:} Regularly scheduled with mentor Zhaoyu Liu, covering various aspects of the research.
    \item \textbf{Literature Review:} Extensive reading on annular modes and their effects on Martian dust activity.
    \item \textbf{Data Analysis:} Using netCDF data formats, developing Python code to read and analyze data, focusing on Mars' annular modes.
    \item \textbf{Collaborations:} Work with Audrey on her research activities, develop independent mini-projects, and engage in group discussions and cross-group presentations.
    \item \textbf{Presentations and Conferences:} Prepare and present at group meetings, submit abstracts for AGU conferences, and participate in workshops and seminars.
    \item \textbf{Reflections and Feedback:} Regular reflections on meetings and research activities, highlighting learning experiences and future plans.
\end{itemize}

\subsection*{Specific Tasks and Deadlines}
\begin{itemize}
    \item May 13, 2022: Understand data formats and vertical structure of Earth's atmosphere.
    \item May 20, 2022: Watch tutorials, engage with research interests, and develop a strategy for collaboration.
    \item May 25, 2022: Explore AGU databases and identify relevant presentations.
    \item June 1, 2022: Review specific papers and recent works by relevant authors.
    \item August 3, 2022: Submit AGU abstract and prepare for upcoming conferences.
    \item September 26, 2022: Weekly meetings to discuss research progress and plan future activities.
\end{itemize}

\section*{GeoFluid Dynamics Workshop essentials}
\subsection*{Summary}
The Geophysical Fluid Dynamics (GFD) Bootcamp is a four-week intensive module held in Summer 2022, led by Assistant Professor Lei Wang. The bootcamp introduces students to key concepts in geophysical fluid dynamics through lectures, historical context, and critical paper reviews.

\subsection*{Objectives and Structure}
\textbf{Goals:} Provide a foundational understanding of Rossby waves, barotropic/baroclinic instability, quasi-geostrophic potential vorticity, and the two-layer atmosphere model.

\textbf{Format:} Bi-weekly meetings focusing on historical development and key equations, supplemented by reading and discussing seminal papers in the field.

\subsection*{Week-by-Week Breakdown}
\textbf{Week One: Rossby Waves}
\begin{itemize}
    \item Key Questions: What are large-scale waves in the atmosphere? What are Rossby waves?
    \item Historical Context: Contributions of Carl-Gustaf Rossby and the Chicago School.
    \item Paper: Rossby et al., 1939, exploring the relationship between zonal circulation intensity and atmospheric action centers.
    \item Activities: Observational studies, derivation of Rossby wave equations.
\end{itemize}

\textbf{Week Two: Potential Vorticity and Baroclinic Instability}
\begin{itemize}
    \item Key Questions: Why do large-scale waves exist? What is baroclinic instability?
    \item Ultimate Goal: Identify a variable that encapsulates flow fields (potential vorticity).
    \item Papers: McIntyre, 2015, and Eady, 1949, on potential vorticity and baroclinic instability.
    \item Activities: Concept discussions, historical development, and derivation of instability models.
\end{itemize}

\textbf{Week Three: Quasi-geostrophic Approximation}
\begin{itemize}
    \item Key Questions: Why simplify atmospheric systems? How to simplify them?
    \item Papers: Charney, 1948, and Vallis, 2016, on the scale of atmospheric motions and the importance of simplification.
    \item Activities: Summarizing key papers, discussing numerical weather forecasting, and understanding quasi-geostrophic scaling.
\end{itemize}

\textbf{Week Four: Two-layer QG Model}
\begin{itemize}
    \item Key Questions: Why is the two-layer QGPV model critical for climate modeling?
    \item Papers: Phillips, 1951, and Held, 2005, on large-scale extratropical flow patterns and the gap between simulation and understanding in climate modeling.
    \item Activities: Deriving quasi-geostrophic PV, discussing model hierarchies, and understanding the role of moisture in models.
\end{itemize}

\section*{Martian Regional Baroclinic Annular Modes (BAM)}
\subsection*{Title}
Regional Features of Periodic Variability in Mars’s Storm Track and Their Relationship with Large Dust Events

\subsection*{Summary}
This document presents a study investigating the periodic variability in Mars's storm tracks and its relationship with large dust events. The study emphasizes the significance of a 20-30 day periodicity in Martian extratropical eddy activity, drawing parallels with Earth's atmospheric behavior.

\subsection*{Key Points}
\textbf{Objective:} Analyze regional features of storm track periodicity using data assimilation datasets and the local finite-amplitude wave activity diagnostic.

\textbf{Methods:} Utilizing EMARS-TES and EMARS-MCS datasets to study the storm tracks and dust lifting mechanisms on Mars.

\textbf{Findings:} Identification of localized midlatitude eddy activity contributing to massive dust storms, with a focus on the southern hemisphere's topographical influence.

\textbf{Implications:} The study aims to provide a better understanding of the mechanics behind dust storm activity on Mars and how these patterns vary

\end{document}